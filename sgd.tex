##Stochastic Gradient Descent (SGD):
   There are a few downsides of the gradient descent algorithm. We need to take a closer look at the amount of computation we make for each iteration of the algorithm.

Say we have 10,000 data points and 10 features. The sum of squared residuals consists of as many terms as there are data points, so 10000 terms in our case. We need to compute the derivative of this function with respect to each of the features, so in effect we will be doing 10000 * 10 = 100,000 computations per iteration. It is common to take 1000 iterations, in effect we have 100,000 * 1000 = 100000000 computations to complete the algorithm. That is pretty much an overhead and hence gradient descent is slow on huge data.

Stochastic gradient descent comes to our rescue !! “Stochastic”, in plain terms means “random”.to explore more:(https://www.geeksforgeeks.org/ml-stochastic-gradient-descent-sgd/).
****“Gradient descent is an iterative algorithm, that starts from a random point on a function and travels down its slope in steps until it reaches the lowest point of that function.”
 This problem is solved by Stochastic GradientDescent. In SGD, it uses only a single sample, i.e., a batch size of one, to perform each iteration. The sample is randomly shuffled and selected for performing the iteration.It is also common to sample a small number of data points instead of just one point at each step and that is called “mini-batch” gradient descent. Mini-batch tries to strike a balance between the goodness of gradient descent and speed of SGD.



In SGD, since only one sample from the dataset is chosen at random for each iteration, the path taken by the algorithm to reach the minima is usually noisier than your typical Gradient Descent algorithm. But that doesn’t matter all that much because the path taken by the algorithm does not matter, as long as we reach the minima and with a significantly shorter training time.

One thing to be noted is that, as SGD is generally noisier than typical Gradient Descent, it usually took a higher number of iterations to reach the minima, because of its randomness in its descent. Even though it requires a higher number of iterations to reach the minima than typical Gradient Descent, it is still computationally much less expensive than typical Gradient Descent. Hence, in most scenarios, SGD is preferred over Batch Gradient Descent for optimizing a learning algorithm.

Pseudocode for SGD in Python: 

def SGD(f, theta0, alpha, num_iters):
    """
       Arguments:
       f -- the function to optimize, it takes a single argument
            and yield two outputs, a cost and the gradient
            with respect to the arguments
       theta0 -- the initial point to start SGD from
       num_iters -- total iterations to run SGD for
       Return:
       theta -- the parameter value after SGD finishes
    """
    start_iter = 0
    theta = theta0
    for iter in xrange(start_iter + 1, num_iters + 1):
        _, grad = f(theta)
  
        # there is NO dot product ! return theta
        theta = theta - (alpha * grad)
This cycle of taking the values and adjusting them based on different parameters in order to reduce the loss function is called back-propagation.
#*#Conclusion
I hope this article was helpful in getting the hang of the algorithm. Stay tuned for more articles to come. Please leave your comments/queries below.

